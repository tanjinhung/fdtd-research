\documentclass[12pt,a4paper,draft]{article}
\usepackage{amsmath}

\begin{document}

\title{Optimizing 3D FDTD with CPML via Temporal Blocking and Parallel Pipelines in FPGA}
\author{Tan Jin Hung}
\maketitle

\begin{abstract}
\end{abstract}

\section{Introduction}
Current age of technology had allowed the use computers to simulate our understanding of the world.
However, such simulations are computational intensive for real-life application.
While the advancement of technology had decrease the time needed to compute simple context simulations.
Complex context of simulations that mimics real-life application still requires an significant amount of time to complete.
One such application of simulation is Finite-Difference Time-Domain (FDTD).

FDTD is a commonly used numerical method for analyzing antennas, design microelectronic devices, and solve other electromagnetic field problems.
This is a type of stencil computation which updates the electric and magnetic field components of a cell in the time domain to compute the electromagnetic function.
This method is derived from Maxwell's equations and discretized by Yee's method.
The main equations that FDTD operated on are Faraday's law and Ampere-Maxwell's law.

\begin{align}
  \nabla \times E &= - \mu \frac{\delta H}{\delta t} \\
  \nabla \times H &= \varepsilon \frac{\delta E}{\delta t} + \sigma E
\end{align}

Equation (1) denotes Faraday's law and is the equation governing the Electric components.
Equation (2) denotes Ampere-Maxwell's law and is the equation govering the Magnetic components.

\section{Background}

\end{document}
