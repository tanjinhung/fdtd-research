\documentclass[12pt,a4paper,draft]{article}
\usepackage{amsmath}
\usepackage[numbers]{natbib}

\begin{document}

\title{Optimizing an FPGA-based 3D FDTD Accelerator through HLS}
\author{Tan Jin Hung}
\maketitle

\begin{abstract}
\end{abstract}

\tableofcontents{}

\section{Introduction}
The current technological era has enabled the use of computers to simulate our complex understanding of the physical world.
However, simulations that accurately mimic real-life phenomena remain computationally intensive for many practical applications.
To manage this complexity, these simulations are often decomposed down into their fundamental physical components.
Among these fundamentals, the simulation of electromagnetic waves remains at the forefront of modern research.

A primary method for simulating electromagnetic waves is the Finite-Difference Time-Domain (FDTD) method. \cite{yee-1138693}
While FDTD is a cornerstone of computational electromagnetics, providing a robust foundation for analysis, it faces significant challenges in three-dimensional space.
Specifically, the iterative nature of the method creates bottleneck in memory bandwidth and data throughput.
Traditionally, these simulations have been offloaded to Graphics Processing Units (GPUs).
However, Field-Programmable Gate Arrays (FPGAs) have emerged as a compelling alternative.
FPGAs offer distinct advantages over GPUs, including deterministic latency, superior energy efficiency, and the ability to implement highly customizable memory architectures for parallelism.

Despite these advantages, FPGAs are traditionally difficult to program using low-level Hardware Description Language (HDLs).
Implementing an FDTD simulation in HDL is tedious and requires the manual design of standard components that are better suited for automation
High-Level Synthesis (HLS) addresses these challenges by allowing designers to describe complex hardware architectures using high-level languages such as C/C++, which is the focus of this paper.

This paper presents an optimized 3D FDTD accelerator designed via HLS, exploring several architectural optimization strategies, including:

\begin{description}
  \item[Spatial Tiling] Maximizing memory bandwidth by utilizing on-chip Block RAMs (BRAMs) to store local stencil and reduce external memory access.
  \item[HLS Directives] Leveraging compiler pragmas to optimize loop pipelining and unrolling, thereby improving the latency of the kernel.
  \item[Task-Level Parallelism] Decomposing the algorithm into concurrent tasks to increase parallelism.
\end{description}

Through these optimization, we demonstrate that an HLS-driven approach can also produce a high-performance FDTD accelerator, while significantly improving code readability, maintainability, and design iteration speed compared to traditional HDL-based workflows.

\section{Background}
This section establishes the theoretical and technical foundation required to design an FPGA-based 3D FDTD accelerator.
It begins by outlining the governing physics of electromagnetic waves propagation via Maxwell's Equations.
Subsequently, it details the transition to the FDTD method by discretizing these equations using Yee's staggered grid and its resulting update equations.
Finally, the section explores the architectural features of FPGAs and the optimization features provided by High-Level Synthesis (HLS) that are essential for high-performance hardware acceleration.

\subsection{Maxwell's Equation}
The underlying physics that governs electromagnetic waves are described by its electric and magnetic properties, which was theorized by J. C. Maxwell. \cite{maxwell-1873}
Maxwell unified the independent laws discovered by Faraday, Amp\`ere, and Gauss, which ties in the field of electric and the field of magnetic, suggesting that they are co-dependent of each other.


\begin{align}
  \nabla \times E &= - \mu \frac{\delta H}{\delta t} - \alpha E\\
  \nabla \times H &= \varepsilon \frac{\delta E}{\delta t} + \sigma E
\end{align}

Equation (1) denotes Faraday-Maxwell's law and is the equation governing the Electric components. \\
Equation (2) denotes Ampere-Maxwell's law and is the equation govering the Magnetic components.

\subsection{The FDTD Method}
\subsubsection{Yee's staggered grid}
\subsubsection{Temporal Update Equation}
\subsection{FPGA \& HLS}

\bibliographystyle{IEEEtranN}
\bibliography{references}
\end{document}
